\documentclass{beamer}
\DeclareFontShape{OT1}{cmss}{b}{n}{<->ssub * cmss/bx/n}{} 
\usetheme{CambridgeUS}
\usecolortheme{rose}
\usepackage{amsmath}
\usepackage{amsfonts}
\usepackage{mathbbol}
\usepackage{xcolor} % before tikz or tkz-euclide if necessary
\usepackage{tkz-euclide} % no need to load TikZ
\usepackage{multirow}
\usepackage{lmodern}
\usepackage{bm}
\usepackage{subcaption}
%\usepackage{subfigure}

\usepackage[
backend=biber,
style=authoryear-icomp,
sortlocale=de_DE,
natbib=true,
url=false, 
doi=true,
eprint=false
]{biblatex}
\addbibresource{../Bibliography/main_ML.bib}



\titlegraphic{\includegraphics[width=2cm]{../Figures/UAMS_RGB.png}
}


\title{Connectomics: Null Hypotheses}
\author{Horacio G\'omez-Acevedo}

\begin{document}

	\begin{frame}[plain]
		\maketitle
	\end{frame}

	\begin{frame}{Today's paper}
		\begin{figure}[h]
			\centering
			%	\begin{subfigure}{0.4\textwidth}
				%		\centering
				\includegraphics[scale=0.45]{../Figures/fig_null_models_paper.png}
				%	\end{subfigure}
		\end{figure}
	\end{frame}
	
	
	\begin{frame}{Main Ideas}
	\begin{itemize}
		\item Rigorous identification and quantification of functionally important features of the brain network architecture.
		\item Using Null models for the presence and relevance of features (sensitivity analysis).
		\item Feature randomization (or bootstraping) to formally investigate relevance.
		\item Multiple null models are possible.
	\end{itemize}
		
	\end{frame}
	
	
	\begin{frame}{Features of Interest}
		Research on imaging across species have "reconstructed" graph organization of brain networks and have determined that:
		\begin{itemize}
			\item Connection profiles
			\item Degree distribution of nodes is heavy tailed ("few hyper-connected nodes")
			\item Densely interconnected network modules
		\end{itemize}
		How to determine the importance of features of interest (FOI)?
		\textbf{Null models} provide a way to systematically compare networks and reveal the presence of a FOI arises as a consequence of other features.
		
		What are network (graph) features?
		\begin{enumerate}
			\item Density, which refers to the proportion of edges.
			\item Degree sequence, which refers to the number of edges incident on each node
			\item Topology 
		\end{enumerate}
		 		
	\end{frame}
	
	\begin{frame}{Null Distribution}
		\begin{figure}[h]
			\centering
			%	\begin{subfigure}{0.4\textwidth}
				%		\centering
				\includegraphics[scale=0.4]{../Figures/fig_null_models_1.png}
				%	\end{subfigure}
		\end{figure}
		
		I
	\end{frame}
	
	\begin{frame}{Graph Null networks}
		We begin with a network feature $F_I$ (e.g., path length) and an empirical network (adjacency matrix).
		
		
		We generate a population of null networks  by randomizing other feature $F_J$ (e.g., topology) and calculating the same feature $F_I$ for each of the randomized variants $F_I^{J_1},F_I^{J_2},\ldots$
		
		Generate the distribution of $\{F_I^{J_k}\}$. The corresponding hypothesis test 
		\begin{equation*}
			H_0\colon F_I=F_I^0 \qquad H_1\colon F_I>F_I^0
		\end{equation*}
		If you calculate the $p$-value associated with the feature $F_I$ distribution. 
		
		"construct a distribution of
		features $\{F_I^{J_k}\}$ under the null hypothesis that the magnitude of feature $F_I^0$ is due
		to properties that were preserved, and not due to properties that were
		randomized (and not preserved)"
		
	\end{frame}
	
	\begin{frame}{Randomization }
		
		A frequently used method is \textit{rewiring} in which pairs of edges are selected at random and then swapped. 
		
		
		\begin{figure}[h]
			\centering
			%	\begin{subfigure}{0.4\textwidth}
				%		\centering
				\includegraphics[scale=0.6]{../Figures/fig_null_models_2.png}
				%	\end{subfigure}
		\end{figure}
	\end{frame}
	
	\begin{frame}{Generative Models}
		These models \textit{build} the network using a predefined wiring rule that embodies the null hypothesis. This process stops when the network has the same size and density as the observed one.
		\begin{figure}[h]
			\centering
			%	\begin{subfigure}{0.4\textwidth}
				%		\centering
				\includegraphics[scale=0.5]{../Figures/fig_null_models_2b.png}
				%	\end{subfigure}
		\end{figure}	
	\end{frame}
	
	\begin{frame}{Other uses of Generative Models}
		
		These models have been also used for \textbf{model identification} or \textbf{model comparison}.
		
		As long as you have a \textbf{cost function}, the generative models produce a way to select an optimal solution.
		
		"This is conceptually similar to
		model identification in formulations of brain networks
		as dynamic systems, such as dynamic causal modelling,
		in which competing accounts of dynamic neural circuit
		interactions are tested to identify the best- fitting or
		most parsimonious model, or when competing hypotheses
		are grouped into distinct families of mechanisms
		or families of models"
		
	\end{frame}
	
	\begin{frame}{Null after null...}
		
		"An important methodological question
		is whether null models uniformly sample the target
		space. The mere fact that a model retains one feature
		and randomizes
		another does not mean that it samples
		the space of all possible realizations exhaustively. "
		\begin{figure}[h]
			\centering
			%	\begin{subfigure}{0.4\textwidth}
				%		\centering
				\includegraphics[scale=0.4]{../Figures/fig_null_models_3.png}
				%	\end{subfigure}
		\end{figure}	
	\end{frame}
\begin{frame}{Null after null... (cont)}	

	"Ultimately, there is no right or wrong null model. The
	null model should be an implementation of an explicit
	and falsifiable null hypothesis that is specific to one’s
	research question."
	
	
\end{frame}
	
	\begin{frame}{Colophon}
		\begin{figure}[h]
			\centering
			%	\begin{subfigure}{0.4\textwidth}
				%		\centering
				\includegraphics[scale=0.35]{../Figures/siam_new.jpg}
				%	\end{subfigure}
		\end{figure}	
	\end{frame}	
\section{Introduction to Hypothesis Testing}	
\begin{frame}{Hypothesis Testing}
	There are key components to take full advantage of th
	
	\begin{itemize}
		\item Likelihood
		\item Statistical Hypothesis
		\item Hypothesis Test
		\item Likelihood Ratio
		\item Asymptotic Properties of the Normal Distribution
		\item $p$-values
	\end{itemize}
\end{frame}
\begin{frame}{Likelihood}
	If we have $X_1,\ldots, X_n$ independent and identically distributed (iid) random variables with a common probability (mass/density) function $f(x;\theta)$ where the parameter $\theta$ is unknown ($\theta \in \Omega$). The likelihood of a sample $\vec{x}=(x_1,\ldots,x_n)$ is 
	\begin{equation*}
		L(\theta,\vec{x})= \prod_{i=1}^n f(x_i,\theta)
	\end{equation*}
	Example: If we have $X_i \sim N(\theta,\sigma^2)$ with $\sigma^2>0$ known but $\theta$ unknown. Then
	\begin{equation*}
		L(\theta,\vec{x})= \left(\frac{1}{2\pi \sigma^2}\right)^{n/2}
		\exp\left(- \frac{1}{2\sigma^2}\sum_{i=1}^n (x_i-\overline{x})^2\right)\exp\left(- \frac{1}{2\sigma^2}n(\overline{x}-\theta)^2\right)
	\end{equation*}
\end{frame}

\begin{frame}{Statistical Hypothesis}
	A Statistical Hypothesis is a conjecture about the probability distribution of a population. 
	
	Example: We suppose that in an experiment we have a random sample from $N(\theta,10)$. 
	\begin{equation*}
		\begin{split}
			H_0 &\colon \text{ The population is } N(5,10) \text{-distributed}\\
			H_1 &\colon \text{ The population is } N(1,10) \text{-distributed}
		\end{split}
	\end{equation*}
\end{frame}

\begin{frame}{Hypothesis Test}
	A Hypothesis Test is a tuple $(X_1,\ldots, X_n; H_0,H_1,G)$, where
	\begin{enumerate}
		\item $(x_1,\ldots, x_n)$ is a sample of $(X_1,\ldots, X_n)$ random variables iid.
		\item $H_0$ and $H_1$ are hypothesis concerning the probability distribution of the population.
		\item $G\subset \mathbb{R}^n$ is Borel set (countable unions of open sets). 
	\end{enumerate}
	The \textbf{level of significance} is defined as 
	\begin{equation*}
		\alpha = P^{H_0}_{X_1,\ldots, X_n}(G)
	\end{equation*}
	We will consider hypothesis such as
	\begin{equation*}
		H_0\colon \theta=\theta_0 (\text{ or }\theta \in \Theta_0) \qquad H_1 \colon \theta\ne \theta_0 (\text{ or }\theta \in \Theta=\Theta_1\cup \Theta_0)
	\end{equation*}
	
\end{frame}

\begin{frame}{Maximum Likelihood Test}
	The \textbf{likelihood ratio} function
	\begin{equation*}
		\Lambda(x_1,\ldots,x_n)= \frac{\sup_{\theta\in \Theta_0}L_\theta(x_1,\ldots,x_n)}{\sup_{\theta\in \Theta} L_\theta(x_1,\ldots, x_n)}
	\end{equation*}
	Let $\widehat{\theta}$ be the maximum likelihood estimate of $\theta$. 
	
	If $\theta_0$ is the true value of $\theta$, then $L(\theta_0)$ is the maximum value of $L(\theta)$. Since $\Lambda \le 1$, then if $H_0$ is true $\Lambda $ should be close to 1, whereas if $H_1$ is true then $\Lambda$ should be smaller. 
	
	We have the decision rule
	\begin{equation*}
		\text{Reject } H_0 \text{ in favor of } H_1 \text{ if }\Lambda \le c,
	\end{equation*}
	where $\alpha= P^{\theta_0}(\Lambda \le c)$.
\end{frame}

\begin{frame}{Asymptotic Properties}
	Under some (regularity)conditions we have the following result:
	
	If the null hypothesis $H_0\colon \theta= \theta_0$.
	\begin{equation*}
		-2 \log \Lambda(X_1,\ldots, X_n) \to \chi^2(1)
	\end{equation*}
	\begin{figure}[h]
		\centering
		%	\begin{subfigure}{0.4\textwidth}
			%		\centering
			\includegraphics[scale=0.45]{../Figures/fig_chi_square1.png}
			%	\end{subfigure}
	\end{figure}
	
	
\end{frame}

\begin{frame}{Asymptotics for the Normal distribution}
	When $\mu$ and $\sigma$ are unknown and testing the hypothesis
	\begin{equation*}
		H_0 \colon \mu=\mu_0 \quad H_1 \colon \mu \ne \mu_0
	\end{equation*}
	The likelihood ratio is given by
	\begin{equation*}
		\Lambda(x_1,\ldots, x_n)= \left\{ 1+ \frac{1}{n-1}\left( \frac{\overline{x}-\mu_0}{s/\sqrt{n}}\right)\right\}^{-n/2}
	\end{equation*}
	where $s^2= \frac{1}{n-1}\sum_{i}(x_i-\overline{x})^2$. The critical regions are of the form
	\begin{equation*}
		G= \{ (x_1,\ldots, x_n)\in \mathbb{R}^n: \left|\frac{\overline{x}-\mu_0}{s/\sqrt{n}} \right|\ge c\}
	\end{equation*}
\end{frame}
\begin{frame}{$p$-values}
	The test statistic $\frac{\overline{x}-\mu_0}{s/\sqrt{n}}$ is critical to reject $H_0$ or not. The decision procedure is as follows
	\begin{equation*}
		\begin{split}
			\text{if }&\left| \frac{\overline{x}-\mu_0}{s/\sqrt{n}} \right| \ge c \text{ then we assume }H_1\\
			\text{if }&\left| \frac{\overline{x}-\mu_0}{s/\sqrt{n}} \right| <c \text{ then we assume }H_0
		\end{split}
	\end{equation*}
	Furthermore, we have the following equivalence
	\begin{equation*}
		P\left(	\left| \frac{\overline{x}-\mu_0}{s/\sqrt{n}} \right| \ge u \right)\le \alpha \iff u \ge c
	\end{equation*}
	The $p$-value associated with the outcome $u$ of the test statistic $\frac{\overline{x}-\mu_0}{S/\sqrt{n}}$ 
	\begin{equation*}
		P\left(	\left| \frac{\overline{x}-\mu_0}{s/\sqrt{n}} \right| \ge |u| \right)
	\end{equation*}
\end{frame}

\begin{frame}{$p$-values (cont)}
	Thus if an outcome $u$ of $\frac{\overline{x}-\mu_0}{S/\sqrt{n}}$  satisfies 
	\begin{equation*}
		P\left(	\left| \frac{\overline{x}-\mu_0}{s/\sqrt{n}} \right| \ge |u| \right) \le \alpha \text{ we accept }H_1
	\end{equation*}
	
\end{frame}
	
	
	
	%\begin{frame}{Fourier Transform}
	%	\begin{figure}[h]
		%	\centering
		%	%	\begin{subfigure}{0.4\textwidth}
			%		%		\centering
			%		\includegraphics[scale=0.65]{../Figures/fig_fourier_transform.}
			%		%	\end{subfigure}
		%\end{figure}
		%\end{frame}
		
		
		%\begin{frame}{References}
		%	Materials and some of the pictures are from \citep{calin}.
		%	\printbibliography 	
		
		%	I have used some of the graphs by hacking TiKz code from StakExchange, Inkscape for more aesthetic plots and other old tricks of \TeX
		
		%\end{frame}
		
		
	\end{document}
